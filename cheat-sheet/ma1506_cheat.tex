\documentclass[10pt,landscape]{article}
\usepackage{multicol}
\usepackage{calc}
\usepackage{ifthen}
\usepackage[landscape]{geometry}
\usepackage{hyperref}
\usepackage{amssymb,amsmath}
\usepackage{amsthm}
\usepackage{graphicx}

% To make this come out properly in landscape mode, do one of the following
% 1.
%  pdflatex latexsheet.tex
%
% 2.
%  latex latexsheet.tex
%  dvips -P pdf  -t landscape latexsheet.dvi
%  ps2pdf latexsheet.ps



% This sets page margins to .5 inch if using letter paper, and to 1cm
% if using A4 paper. (This probably isn't strictly necessary.)
% If using another size paper, use default 1cm margins.
\ifthenelse{\lengthtest { \paperwidth = 11in}}
	{ \geometry{top=.4in,left=.4in,right=.4in,bottom=.4in} }
	{\ifthenelse{ \lengthtest{ \paperwidth = 297mm}}
		{\geometry{top=0.5cm,left=0.5cm,right=0.5cm,bottom=0.5cm} }
		{\geometry{top=0.5cm,left=0.5cm,right=0.5cm,bottom=0.5cm} }
	}

% Turn off header and footer
\pagestyle{empty}

% Change hyperlinks to something slightly more elegant
\hypersetup{
	colorlinks = true,
	linkcolor=blue,
} 

% Set up remark environment with amsthm
\theoremstyle{remark}
\newtheorem*{remark}{Remark}
\newtheorem*{theorem}{Theorem}

% set up some new operators with amsmath
\DeclareMathOperator{\Lapl}{\mathcal{L}}

% Redefine section commands to use less space
\makeatletter
\renewcommand{\section}{\@startsection{section}{1}{0mm}%
                                {-1ex plus -.5ex minus -.2ex}%
                                {0.5ex plus .2ex}%x
                                {\normalfont\large\bfseries}}
\renewcommand{\subsection}{\@startsection{subsection}{2}{0mm}%
                                {-1explus -.5ex minus -.2ex}%
                                {0.5ex plus .2ex}%
                                {\normalfont\normalsize\bfseries}}
\renewcommand{\subsubsection}{\@startsection{subsubsection}{3}{0mm}%
                                {-1ex plus -.5ex minus -.2ex}%
                                {1ex plus .2ex}%
                                {\normalfont\small\bfseries}}
\makeatother

% Define BibTeX command
\def\BibTeX{{\rm B\kern-.05em{\sc i\kern-.025em b}\kern-.08em
    T\kern-.1667em\lower.7ex\hbox{E}\kern-.125emX}}

% Don't print section numbers
\setcounter{secnumdepth}{0}


\setlength{\parindent}{0pt}
\setlength{\parskip}{0pt plus 0.5ex}


% -----------------------------------------------------------------------

\begin{document}

\raggedright
\footnotesize
\begin{multicols}{3}


% multicol parameters
% These lengths are set only within the two main columns
%\setlength{\columnseprule}{0.25pt}
\setlength{\premulticols}{1pt}
\setlength{\postmulticols}{1pt}
\setlength{\multicolsep}{1pt}
\setlength{\columnsep}{2pt}

\begin{center}
     \Large{\textbf{$\mathbb{MA}1506$ Cheat Sheet}} \\
\end{center}

\section{Ordinary Differential Equations}
\subsection{Linear First-Order ODE}
The standard form for linear first-order ODEs are:
\begin{equation}
	\frac{dy}{dx} + p(x)y = r(x)
\end{equation}

The integrating factor $\mu(t)$ is given by
\begin{equation}
	\mu(t) = e^{\int p(x) dx}
\end{equation}

\subsection{Bernoulli Equations}
Bernoulli equations have the standard form:
\begin{equation}
	y' + p(x)y = q(x)y^n, n\in\mathbb{R}
	\label{eqn:bernoulli}
\end{equation}

When $n=0,1$, the equation is linear and we can solve it using the integrating factor.
However, for other values of $n$, it is necessary to reduce the equation to linear form.

First, divide Equation \ref{eqn:bernoulli} by $y^n$.
We will use the substitution $v=y^{1-n}$, such that the derivative $\frac{dv}{dx} = (1-n)y^{-n}y'$.
We then obtain the following:
\begin{equation}
	\frac{1}{1-n}v' + p(x)v = q(x)
\end{equation}

\subsection{Second-Order ODE}
The standard form for second-order ODEs is:
\begin{equation}
	\frac{d^2y}{dx^2} + p(x)\frac{dy}{dy} + q(x)y = F(x)
\end{equation}

If $F(x) \equiv 0$, the equation is \emph{homogenous}. Otherwise, it is \emph{nonhomogenous}.

A solution of a second-order ODE on some interval $I$ is a function $y=h(x)$ with derivatives $y'=h'(x)$ and $y''=h''(x)$ satisfying the ODE $\forall x$ in $I$.

\subsubsection{Homogeous Second-Order ODEs}
\begin{equation}
	\frac{d^2y}{dx^2} + p(x)\frac{dy}{dy} + q(x)y = 0
	\label{eqn:homo_so_ode}
\end{equation}

Any \underline{linear} combination of two solutions for Equation \ref{eqn:homo_so_ode} on an open interval $I$ is also a solution on $I$, i.e. sums and constant multiples of solutions are also themselves solutions.

\begin{remark}
	The above is not true for nonhomogeous equations.
\end{remark}

As it is a solution to $y' + ky = 0, k\in\mathbb{R}$, we thus find that $y=e^{\lambda x}$ is also a solution to Equation \ref{eqn:homo_so_ode} if $\lambda$ is a solution to Equation \ref{eqn:characteristic}.

\begin{equation}
	\lambda^2+a\lambda+b = 0
	\label{eqn:characteristic}
\end{equation}

\subsubsection{Case 1: two real solutions $\lambda_1$ and $\lambda_2$}
In this case, the solutions to the ODE are $e^{\lambda_1 x}$ and $e^{\lambda_2 x}$ which are linearly independent solutions on any interval.

The corresponding general solution is thus given by
\begin{equation}
	y=c_1 e^{\lambda_1 x} + c_2 e^{\lambda_2 x}
\end{equation}

\subsubsection{Case 2: one real solution $\lambda$}
The corresponding general solution is given by
\begin{equation}
	y = c_1 e^{\lambda x} + c_2 x e^{\lambda x}
\end{equation}

\subsubsection{Case 3: complex solutions $\lambda \pm \mu i$}
The corresponding general solution is given by
\begin{equation}
	y = c_1 e^{\lambda x}\cos{\mu x} + c_2e^{\lambda x}\sin{\mu x}
\end{equation}

\subsubsection{Nonhomogenous Second-Order ODEs}
\begin{theorem}
	The general solution of the nonhomogenous differential equation $p(x)y'' + q(x)y' + r(x)y = G(x)$ can be written as:
	$$ y(x) = y_p(x) + y_c(x) $$
	where $y_p(x)$ is a particular solution of Equation $p(x)y''+q(x)y'+r(x)y=G(x)$ and $y_c(x)$ is the general solution of the \emph{complementary} equation $p(x)y''+b'+r(x)y=0$.
\end{theorem}

\subsubsection{Method of Undetermined Coefficients}
$$y'' + p(x)y' + q(x)y = r(x)$$
Guess Table
\resizebox{\linewidth}{!}{%
\begin{tabular}{@{}rl@{}}
	\hline
	$r(x)$			& Guess		\\\hline
	$k\in\mathbb{R}$& $A$ \\
	$5x+7$			& $Ax+B$ \\
	$3x^2-2$		& $Ax^2+Bx+C$ \\
	$\sin{4x}$		& $A\cos{4x}+B\sin{4x}$ \\
	$\cos{4x}$		& $A\cos{4x}+B\sin{4x}$ \\
	$e^{5x}$		& $Ae^{5x}$ \\
	$(9x-2)e^{5x}$	& $(Ax+B)e^{5x}$ \\
	$x^2e^{5x}$		& $(Ax^2+Bx+C)e^{5x}$ \\
	$e^{3x}\sin{4x}$& $Ae^{3x}\cos{4x} + Be^{3x}\sin{4x}$ \\
	$5x^2\sin{4x}$	& $(Ax^2+Bx+C)\cos{4x} + (Ex^2+Fx+G)\sin{4x}$ \\
	$xe^{3x}\cos{4x}$& $(Ax+B)e^{3x}\cos{4x} + (Cx+E)e^{3x}\sin{4x}$ \\
	$(5x+7)+\sin{4x}$& $(Ax+B)+(C\cos{4x}+D\sin{4x})$ \\\hline
\end{tabular}}

\subsubsection{Variation of Parameters}
$$y'' + p(x)y' + q(x)y = r(x)$$
As above, the theorem for the general solution holds and we find the complementary solution $y_c(x) = c_1y_1(x) + c_2y_2(x)$.

Now let us define a pair of functions $u(x)$ and $v(x)$ such that
\begin{equation}
	y_p(x) = u(x)y_1(x) + v(x)y_2(x)
\end{equation}

\begin{equation}
	u'(x)y_1(x) + v'(x)y_2(x) = 0
	\label{eqn:variation1}
\end{equation}

\begin{align}
	y_p'(x) 	& = u(x)y_1'(x) + v(x)y_2'(x) \\
	y_p''(x) 	& = u'(x)y_1'(x) + u(x)y_1'(x) + v'(x)y_2'(x) + v(x)y_2''(x)
\end{align}

\begin{align}
	u'(x) & = -\frac{y_2(x)r(x)}{y_1(x)y_2'(x) - y_1'(x)y_2(x)}\\
	v'(x) & = \frac{y_1(x)r(x)}{y_1(x)y_2'(x) - y_1'(x)y_2(x)}\\
\end{align}

Thereafter, we can obtain $u(x)$ and $v(x)$ by integration.
The constant of integration in $u(x)$ and $v(x)$ can be ignored.

\section{Mathematical Modelling}
\subsection{Euler's Bending Equation}
Suppose a cantilevered beam with a Young's modulus $E$, a distributed load across its length $w(x)$ and a deflection $v(x)$ as functions of horizontal position.
\begin{equation}
	\frac{d^2}{dx^2}\Big[ EI_z \frac{d^2v}{dx^2}  \Big] = w(x)
\end{equation}

\subsection{Malthus Model of Population}
For a population of initial size $P_0$ with size $P(t)$ at time $t$, and a population growth rate $r = \text{birth rate} - \text{death rate}$,
\begin{equation}
	P(t) = P_0 e^{rt}
\end{equation}
\begin{tabular}{rl}
	\hline
	$r < 0$ & population collapse (more deaths than births per capita) \\\hline
	$r = 0$ & remains stable (if and only if) \\\hline
	$r > 0$ & population explosion (more births than deaths per capita)\\\hline
\end{tabular}

\section{Laplace Transform}
Let $f$ be a function defined for $t \geq 0$.
The Laplace transform of $f$ is the function $F(s)$, where
\begin{equation}
	F(s) = \Lapl(f) = \int_0^\infty e^{-st} f(t)\: dt
\end{equation}

\begin{theorem}
	For some $a,b\in\mathbb{R}$,
	$$\Lapl(af(t)+bg(t)) = a\Lapl(f) + b\Lapl(g)$$

	This is also true for the inverse Laplace transform.
\end{theorem}

\begin{remark}
The Laplace transform is independent of whether the target function is continuous or not.
\end{remark}

\begin{theorem}
	Suppose the continuous function $f(t)$ has a well-defined Laplace transform on $[0,\infty)$ and $f'(t)$ is piecewise-continuous on $[0,\infty)$.
	Thus, $\Lapl(f'(t))$ exists, and the following is true for $s>a$:
	$$\Lapl(f'(t)) = s\Lapl(f) - f(0)$$
\end{theorem}

Some useful identities:
\begin{align}
	\Lapl(\cos{\omega t}) = \frac{s}{s^2+\omega^2} \qquad \Lapl(\sin{\omega t}) = \frac{\omega}{s^2+\omega^2} 
\end{align}

\section{Linear Algebra}

\section{System of ODEs}

\section{Partial Differential Equations}

\section{Trigonometric Identities}
\subsection{Hyperbolic Functions}
$$\sinh{x} = \frac{e^x-e^{-x}}{2} = \frac{e^{2x}-1}{2e^x} = \frac{1-e^{-2x}}{2e^{-x}}$$
$$\cosh{x} = \frac{e^x + e^{-x}}{2} = \frac{e^{2x}+1}{2e^x} = \frac{1+e^{-2x}}{2e^{-x}}$$
$$\tanh{x} = \frac{\sinh{x}}{\cosh{x}} = \frac{e^x-e^{-x}}{e^x+e^{-x}}$$

\subsection{Pythagorean Identities}
$$\sin^2{u} + \cos^2{u} = 1$$
$$1+\tan^2{u} = \sec^2{u}$$
$$1+\cot^2{u} = \csc^2{u}$$

\subsection{Sum and Difference Formulae}
$$\sin(u\pm v) = \sin{u}\cos{v} \pm \cos{u}\sin{v}$$
$$\cos(u\pm v) = \cos{u}\cos{v} \mp \sin{u}\sin{v}$$
$$\tan(u\pm v) = \frac{\tan{u}\pm\tan{v}}{1\mp\tan{u}\tan{v}}$$

\subsection{Double Angle Formulae}
$$\sin{2u} = 2\sin{u}\cos{u}$$
$$\cos{2u} = \cos^2{u} - \sin^2{u} = 2\cos^2{u} - 1 = 1-2\sin^2{u}$$
$$\tan{2u} = \frac{2\tan{u}}{1-\tan^2{u}}$$

\subsection{Half Angle Identities}
$$\sin^2{u} = \frac{1-\cos{2u}}{2}$$
$$\cos^2{u} = \frac{1+\cos{2u}}{2}$$
$$\tan^2{u} = \frac{1-\cos{2u}}{1+\cos{2u}}$$

\subsection{Sum $\to$ Product Identities}
$$\sin{u} + \sin{v} = 2\sin{\frac{u+v}{2}}\cos{\frac{u-v}{2}}$$
$$\sin{u} - \sin{v} = 2\cos{\frac{u+v}{2}}\sin{\frac{u-v}{2}}$$
$$\cos{u} + \cos{v} = 2\cos{\frac{u+v}{2}}\cos{\frac{u-v}{2}}$$
$$\cos{u} - \cos{v} = -2\sin{\frac{u+v}{2}}\sin{\frac{u-v}{2}}$$

\subsection{Product $\to$ Sum Identities}
$$\sin{u}\sin{v} = \frac{1}{2} [\cos{(u-v)} - \cos{(u+v)}]$$
$$\cos{u}\cos{v} = \frac{1}{2} [\cos{(u-v)} + \cos{(u+v)}]$$
$$\sin{u}\cos{v} = \frac{1}{2} [\sin{(u+v)} + \sin{(u-v)}]$$
$$\cos{u}\sin{v} = \frac{1}{2} [\sin{(u+v)} - \sin{(u-v)}]$$

\subsection{Parity Identities}
$$\sin{(-u)} = -\sin{u}$$
$$\cos{(-u)} = \cos{u}$$
$$\tan{(-u)} = -\tan{u}$$

$$\cot{(-u)} = -\cot{u}$$
$$\csc{(-u)} = -\csc{u}$$
$$\sec{(-u)} = \sec{u}$$

{\hfill \texttt{darren.wee@u.nus.edu}}

{\hfill \texttt{Updated \today.}}
\end{multicols}
\end{document}
